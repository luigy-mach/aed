\documentclass[a4paper,11pt]{article}
\usepackage[activeacute,spanish]{babel} %Para el idioma espa�ol
\usepackage[latin1]{inputenc} %para incluir e�e, puntuaci�n
\usepackage{anysize} %margenes
\usepackage{listings} %para incluir archivos c++ y otros
\usepackage{color} %para incluir colores
\marginsize{1.5cm}{1.5cm}{0.5cm}{0.5cm}
%margenes izquierda, derecha, superior, inferior respectivamente

\begin{document}
\definecolor{OliveGreen}{cmyk}{0.64,0,0.95,0.40} %definiendo un nuevo color

\begin{center}
\textbf{Comentando Listas Template Simplemente Enlazada\\Jos\'e Miguel Huam\'an Cruz, C.U.I.:20103389,
josemiki24@gmail.com\\
Escuela Profesional de Ciencia de la Computaci�n\\
Facultad de Producci�n y Servicios\\
Universidad Nacional de San Agust�n\\}
\end{center}

\lstset{numbers=left, numberstyle=\tiny} %enumerado del c�digo en la parte izquierda

%\lstset{backgroundcolor=\color{yellow}} %color de fondo

\lstset{frameround=tttt}
%esquina ovaladas derecha superior, derecha inferior, izquierda inferior, izquierda superior, respectivamente

\lstset{keywordstyle=\color{blue},commentstyle=\color{OliveGreen}}
%color de las palabras clave y delos comentarios

\lstset{showstringspaces=false} %para no mostrar la separaci�n de cadena

\section{Comentando Codigo "ListaT"}
\lstinputlisting[language=C++,frame={btrl}]{lista.cpp}
\end{document}
